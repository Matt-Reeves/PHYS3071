\documentclass[11pt]{article}
\usepackage[left=2cm,right=2cm,bottom=2cm,top =1cm]{geometry}                % See geometry.pdf to learn the layout options. There are lots.
\usepackage{amsmath}
\geometry{letterpaper}   
\usepackage[symbol]{footmisc}
\renewcommand{\thefootnote}{\fnsymbol{footnote}}
                % ... or a4paper or a5paper or ... 
%\geometry{landscape}                % Activate for for rotated page geometry
%\usepackage[parfill]{parskip}    % Activate to begin paragraphs with an empty line rather than an indent

\usepackage{graphicx}
\usepackage{amssymb}
\usepackage{epstopdf}
\DeclareGraphicsRule{.tif}{png}{.png}{`convert #1 `dirname #1`/`basename #1 .tif`.png}

\usepackage[most]{tcolorbox}
\usepackage{xcolor}
\definecolor{vlightgray}{rgb}{0.9,0.9,0.9}

\title{Workshop \#: Relaxation Methods for Boundary Value Problems \\ Possion's Equation }
\author{\textbf{Reading: } Numerical Recipes, Ch 20.5 }
\date{}                                           % Activate to display a given date or no date

\begin{document}
\maketitle

 Consider the boundary value problem defined by the Poisson equation in the unit square
\begin{align}
-\nabla^2 \phi &= f(x,y), &  \Omega &= \{x,y\} \in [0,1] & \phi &= 0, \quad \{x,y\} \in \partial \Omega.
\end{align}
As a test problem for relaxation methods, consider the source term
\begin{equation}
f(x,y) = [2 + \pi^2(1-y)y] \sin{\pi x} + [2 + \pi^2(1-x)x] \sin{\pi y},
\end{equation}
which admits the exact solution
\begin{equation}
\phi(x,y) = y(1-y) \sin{\pi x} + x(1-x) \sin{\pi y}.
\end{equation}
\dotfill
\begin{enumerate}
\item Write a program to calculate the solution via the Jacobi method:
\begin{equation}
u_{i,j}^{(k+1)} = 	\frac{1}{4}\left( u_{i+1,j}^{(k)} + u_{i-1,j}^{(k)} + u_{i,j+1}^{(k)} + u_{i,j-1}^{(k)} + \Delta^2 f_{i,j}\right).
\end{equation}
Evaluate the sum-of-squares error vs. the number of iterations. 
\item Do the same for the Gauss-Seidel method:
\begin{equation}
u_{i,j}^{(k+1)}  = \frac{1}{4}\left( u_{i+1,j}^{(k)} + u_{i-1,j}^{(k+1)} + u_{i,j+1}^{(k)} + u_{i,j-1}^{(k+1)} + \Delta^2 f_{i,j}\right).
\end{equation}
\item Now try Successive Over-Relaxation (SOR):
\begin{equation}
u_{i,j}^{(k+1)}  = u_{i,j}^{(k)} -\omega \frac{\xi_{i,j}}{4},
\end{equation}
where the residual $\xi_{i,j} = u_{i+1,j} + u_{i-1,j} + u_{i,j+1} + u_{i,j-1} - 4 u_{-i,j} + \Delta^2 f_{i,j}$ and $1 < \omega < 2$. Compare a couple of  suboptimal values of $\omega$ with the ``optimal" value  $\omega \simeq 2 / ( 1 + \pi /J)$, where $J$ is the number of grid points in each dimension.
\item Verify the number of iterations $r$, required to reduce the error by $10^{-p}$, scales as expected for each method:
\begin{align}
r & \simeq \tfrac{1}{2} p J^2 \quad \textrm{(Jacobi)} & r & \simeq \tfrac{1}{4} p J^2 \quad \textrm{(Gauss-Seidel)} &  r & \simeq \tfrac{1}{3} p J \quad \textrm{(SOR)}
\end{align}  
\item\footnote[2]{Optional}  Improve the SOR method by implementing \emph{odd-even ordering} on the mesh-points, and \emph{Chebyshev acceleration} on the relaxation parameter:
\begin{align}
\omega^{(0)} &= 1 \nonumber \\
\omega^{(1/2)} &= 1/(1- \rho_{\rm{Jacobi}}^2/2)  \\
\omega^{(n+ 1/2)} &= 1/(1- \rho_{\rm{Jacobi}}^2 \omega^{(n)}/4); \quad n = 1/2,1,\dots   \nonumber
\end{align}
\item $^\dagger$ Direct matrix methods, although impractical for large problems, can be useful for small or medium sized problems. An example direct matrix code is provided in \fcolorbox{white}{vlightgray}{\texttt{\color{magenta}{directMatrixPoisson.cpp}}}, which makes use of the C++ linear algebra library \fcolorbox{white}{vlightgray}{\texttt{\color{magenta}{Eigen}}} to solve this system with sparse matrices and LU factorization.  Compare how the speed and accuracy of the direct matrix method scales with $J$ compared to a relaxation method. 
\end{enumerate}



\end{document}  